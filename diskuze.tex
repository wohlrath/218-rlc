\section*{Diskuze}
Měření periody i aperiodizačních odporů jsme prováděli s touž cívkou, takže pokud by byla obě měření správná, měly by se hodnoty $L_1=\SI{0.52(5)}{\henry}$ a $L_2=\SI{0.34(2)}{\henry}$ shodovat, což v našem případě platí jen řádově.


Měření aperiodizačních odporů bylo zatíženo velkou systematickou chybou, protože záleželo na tom, jestli daný průběh vyhodnotím jako periodický či aperiodický.
Pokud je hodnota $L_1$ správná, znamenalo by to, že je buď měření zatíženo nějakou jinou neznámou systematickou chybou, nebo jsme hodnoty $R_{ap}$ určili přibližně \num{1.2} krát menší.
Signál byl zatížen nezanedbatelným šumem, takže je možné, že druhý kmit již byl pod hranicí šumu, a proto nebylo možné ho pozorovat a došlo tak k mylnému označení průběhu za aperiodický.
Z těchto důvodů považuji hodnotu $L_1$ za přesnější.

Při fitování průběhu proudu RC obvodem se u všech měření stalo to, že hodnoty na kraji měřeného časového intervalu ležely znatelně pod proloženou křivkou, zatímco hodnoty ve středu ležely nad ní.
Průběh tedy nebyl přesně exponenciální.
Nicméně naměřené relaxační doby se velmi dobře shodují s teoretickými, žádná se neliší o více než \SI{5}{\percent}.