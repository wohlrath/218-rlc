\documentclass[a4paper]{article}

\usepackage[czech]{babel} %https://github.com/michal-h21/biblatex-iso690
\usepackage[
   backend=biber      % if we want unicode 
  ,style=iso-numeric % or iso-numeric for numeric citation method          
  ,babel=other        % to support multiple languages in bibliography
  ,sortlocale=cs_CZ   % locale of main language, it is for sorting
  ,bibencoding=UTF8   % this is necessary only if bibliography file is in different encoding than main document
]{biblatex}

\usepackage[utf8]{inputenc}
\usepackage{fancyhdr}
\usepackage{amsmath}
\usepackage{amssymb}
\usepackage[left=2cm,right=2cm,top=2.5cm,bottom=2.5cm]{geometry}
\usepackage{graphicx}
\usepackage{pdfpages}
\usepackage{url}
\usepackage{multirow}

\usepackage{siunitx}
\sisetup{locale = DE, separate-uncertainty = true} %   kdybych chtel +/-

\usepackage{float}
\newfloat{graph}{htbp}{grp}
\floatname{graph}{Graf}
\newfloat{tabulka}{htbp}{tbl}
\floatname{tabulka}{Tabulka}

\renewcommand{\thefootnote}{\roman{footnote}}

\pagestyle{fancy}
\lhead{Praktikum II - (18) Přechodové jevy v RLC obvodu}
\rhead{Vladislav Wohlrath}
\author{Vladislav Wohlrath}

\bibliography{source}

\begin{document}

\begin{titlepage}
\includepdf[pages={1}]{./graficos/titlelist.pdf}
\end{titlepage}

\section*{Pracovní úkoly}
\begin{enumerate}
\item Sestavte obvod podle obr. 1 (viz studijní text k úloze) a změřte pro obvod v periodickém stavu závislost doby kmitu $T$ na velikosti zařazené kapacity alespoň pro pět hodnot z intervalu ($C$=\num{0.1}--\SI{10}{\micro\farad}, $R$ = \SI{20}{\ohm}). Výsledky měření zpracujte graficky a vyhodnoťte velikost indukčnosti $L$ zařazené v obvodu.
\item Stanovte hodnoty aperiodizačních odporů pro pět hodnot kapacit zařazeného kondenzátoru. I v tomto případě stanovte velikost indukčnosti $L$.
\item Změřte závislost relaxační doby seriového obvodu RC podle obr. 2 (viz. studnijní text k úloze) na velikosti odporu a na velikosti kapacity v obvodu. Výsledky měření zpracujte graficky a porovnejte s teoretickými.
\end{enumerate}

%Teoretická část
\section*{Teoretická část}
Pokud se v sériovém RLC obvodu změní v čase $t=0$ přivedené napětí z nulové hodnoty na hodnotu $\varepsilon$, pak má časový průběh proudu v sériovém RLC obvodu různý charakter v závislosti na hodnotách $R$, $L$ a $C$.
Označíme
\begin{equation*}
A=\frac{R}{2L} \,, \qquad  B^2=\frac{1}{LC} - \frac{R^2}{4L^2} \,.
\end{equation*}

Potom je průběh proudu podle \cite{skripta}
\begin{enumerate}
\item pro $A^2<1/LC$: periodické --- $I(t) = (\varepsilon/BL) \exp(-At) \sin(Bt)$
\item pro $A^2=1/LC$: mezní aperiodické --- $I(t) = (\varepsilon/L)t \exp(-At)$
\item pro $A^2>1/LC$: aperiodické --- $I(t) = (\varepsilon/BL) \exp(-At) \sinh(Bt)$
\end{enumerate}

Při periodickém pohybu odpovídá konstanta $B$ kruhové frekvenci kmitů.
Pokud platí $1/LC \gg (R/L)^2$, což je v našem případě dobře splněno, můžeme indukčnost $L$ vyjádřit z periody kmitů
\begin{equation} \label{e:Lper}
L=\frac{T^2}{4\pi^2 C} \,.
\end{equation}

Při mezním aperiodickém pohybu můžeme podmínku $A^2=1/LC$ upravit na $R_{ap}=2 \sqrt{L/C}$.
Pokud tedy správně určíme aperiodizační odpor $R_{ap}$, můžeme vypočítat indukčnost $L$ podle
\begin{equation} \label{e:Laper}
L=\frac{R_{ap}^2C}{4} 	,.
\end{equation}

V sériovém RC obvodu po vypnutí zdroje klesá proud s časem úměrně funkci $\exp(-t/\tau)$, kde
\begin{equation} \label{e:tau}
\tau=RC \,,
\end{equation}
je \emph{relaxační doba}.

%Výsledky měření
\section*{Výsledky měření}
Měření proběhlo při pokojové teplotě ($\approx \SI{22}{\degreeCelsius}$).
Všechny uvedené odchylky jsou standardní.

\subsection*{Tlumené kmity}

Změřili jsme dobu kmitu $T$ při $R=\SI{20}{\ohm}$ v závislosti na $C$.
Výsledky jsou uvedeny v tabulce \ref{t:per} a grafu \ref{g:per}.
Měřili jsme vždy dobu 5--7 kmitů a měření jsme provedli vícekrát, vzhledem k obdrženým výsledkům odhadujeme chybu měření periody na přibližně \SI{0.5}{\percent}.
Pro každou kapacitu uvádíme též indukčnost počítanou ze vzorce \eqref{e:Lper}.
Chybu odporové a kapacitní dekády zanedbáváme a metodou přenosu chyb jsme určili chybu indukčnosti \SI{1}{\percent}.

\begin{tabulka}[htbp]
\centering
\begin{tabular}{c|c|c}
$C$ (\si{\micro\farad}) & $T$ (\si{\milli\second}) & $L$ (\si{\henry}) \\ \hline
\num{0.1} & \num{1.36(1)} & \num{0.469(5)} \\
\num{0.4} & \num{2.82(2)} & \num{0.504(6)} \\
\num{1} & \num{4.34(3)} & \num{0.478(5)} \\
\num{3.2} & \num{8.26(5)} & \num{0.540(6)} \\
\num{10} & \num{15.2(1)} & \num{0.587(6)} \\
\end{tabular}
\caption{Doba kmitu v periodickém stavu v závislosti na $C$ pro $R=\SI{20}{\ohm}$}
\label{t:per}
\end{tabulka}

\begin{graph}[htbp] 
\centering
\input{per.tex}
\caption{Doba kmitu v periodickém stavu v závislosti na $C$ pro $R=\SI{20}{\ohm}$}
\label{g:per}
\end{graph}

Hodnoty $L$ v tabulce \ref{t:per} jsme statisticky zpracovali a dostáváme indukčnost cívky $L_1=\SI{0.52(5)}{\henry}$.


\subsection*{Aperiodizační odpor}

Měřili jsme aperiodizační odpor $R_{ap}$ v závislosti na kapacitě $C$.
Chybu aperiodizačního odporu jsme odhadovali tak, abychom pokryli interval, ve kterém jsme nedokázali rozlišit, jestli je průběh proudu periodický či aperiodický.
Uvádíme též hodnotu indukčnosti $L$ vypočtené pro každé měření podle \eqref{e:Laper}.
Odchylku indukčnosti určujeme metodou přenosu chyb jako dvojnásobek relativní chyby aperiodizačního odporu.
Výsledky jsou v tabulce \ref{t:aper} a v grafu \ref{g:aper}.


\begin{tabulka}[htbp]
\centering
\begin{tabular}{c|c|c}
$C$ (\si{\micro\farad}) & $R_{ap}$ (\si{\ohm}) & $L$ (\si{\henry}) \\ \hline
\num{1} & \num{1150(30)} & \num{0.33(2)} \\
\num{3} & \num{665(10)} & \num{0.33(1)} \\
\num{6} & \num{480(10)} & \num{0.35(2)} \\
\num{8} & \num{430(10)} & \num{0.37(2)} \\
\num{10} & \num{370(10)} & \num{0.34(2)} \\
\end{tabular}
\caption{Aperiodizační odpor $R_{ap}$ v závislosti na $C$}
\label{t:aper}
\end{tabulka}

\begin{graph}[htbp] 
\centering
\input{aper.tex}
\caption{Aperiodizační odpor $R_{ap}$ v závislosti na $C$}
\label{g:aper}
\end{graph}

Hodnoty $L$ v tabulce \ref{t:aper} jsme statisticky zpracovali a dostáváme indukčnost cívky $L_2=\SI{0.34(2)}{\henry}$.

\subsection*{RC obvod}

Měřili jsme závislost relaxační doby seriového RC obvodu v závislosti na odporu $R$ a kapacitě $C$.
Změřili jsme několik kapacit pro $R=\SI{1}{\kilo\ohm}$ a pro několik odporů pro $C=\SI{1}{\micro\farad}$.
Programem ISES jsme závislost odporu na čase proložili funkcí ve tvaru $a\cdot\exp(bt)$.
Porovnáním s definičním vztahem relaxační doby získáváme
\begin{equation}
\tau=\frac{-1}{b} \,.
\end{equation}

Výsledky jsou v tabulce \ref{t:RC} a v grafech \ref{g:RCr} a \ref{g:RCc}.
Uvádíme jen hodnoty $\tau$, přímo hodnoty $b$ neuvádíme, případně jsou k nahlédnutí v záznamu z měření.
Kromě změřené hodnoty $\tau$ uvádíme také teoretickou hodnotu $\tau_t$ podle \eqref{e:tau}.

\begin{tabulka}[htbp]
\centering
\begin{tabular}{c|c|c|c}
$C$ (\si{\micro\farad}) & $R$ (\si{\kilo\ohm}) & $\tau$ (\si{\milli\second}) & $\tau_t$ (\si{\milli\second}) \\ \hline
\num{1} & \num{50} & \num{49.2} & \num{50} \\
\num{1} & \num{10} & \num{10.0} & \num{10} \\
\num{1} & \num{4} & \num{3.97} & \num{4} \\
\num{1} & \num{2} & \num{1.96} & \num{2} \\
\num{1} & \num{0.5} & \num{0.514} & \num{0.5} \\
\num{1} & \num{1} & \num{0.962} & \num{1} \\
\num{0.5} & \num{1} & \num{0.523} & \num{0.5} \\
\num{2} & \num{1} & \num{1.92} & \num{2} \\
\num{4} & \num{1} & \num{3.82} & \num{4} \\
\num{6} & \num{1} & \num{5.77} & \num{6} \\
\num{10} & \num{1} & \num{9.60} & \num{10} \\
\end{tabular}
\caption{Závislost relaxační doby $\tau$ na $R$ a $C$}
\label{t:RC}
\end{tabulka}

\begin{graph}[htbp] 
\centering
\input{RCr.tex}
\caption{Závislost relaxační doby $\tau$ na $C$ pro $R=\SI{1}{\kilo\ohm}$ (pro přehlednost logaritmické měřítko)}
\label{g:RCr}
\end{graph}

\begin{graph}[htbp] 
\centering
\input{RCc.tex}
\caption{Závislost relaxační doby $\tau$ na $R$ pro $C=\SI{1}{\micro\farad}$ (pro přehlednost logaritmické měřítko)}
\label{g:RCc}
\end{graph}

%Diskuze výsledků
\section*{Diskuze}
Měření periody i aperiodizačních odporů jsme prováděli s touž cívkou, takže pokud by byla obě měření správná, měly by se hodnoty $L_1=\SI{0.52(5)}{\henry}$ a $L_2=\SI{0.34(2)}{\henry}$ shodovat, což v našem případě platí jen řádově.


Měření aperiodizačních odporů bylo zatíženo velkou systematickou chybou, protože záleželo na tom, jestli daný průběh vyhodnotím jako periodický či aperiodický.
Pokud je hodnota $L_1$ správná, znamenalo by to, že je buď měření zatíženo nějakou jinou neznámou systematickou chybou, nebo jsme hodnoty $R_{ap}$ určili přibližně \num{1.2} krát menší.
Signál byl zatížen nezanedbatelným šumem, takže je možné, že druhý kmit již byl pod hranicí šumu, a proto nebylo možné ho pozorovat a došlo tak k mylnému označení průběhu za aperiodický.
Z těchto důvodů považuji hodnotu $L_1$ za přesnější.

Při fitování průběhu proudu RC obvodem se u všech měření stalo to, že hodnoty na kraji měřeného časového intervalu ležely znatelně pod proloženou křivkou, zatímco hodnoty ve středu ležely nad ní.
Průběh tedy nebyl přesně exponenciální.
Nicméně naměřené relaxační doby se velmi dobře shodují s teoretickými, žádná se neliší o více než \SI{5}{\percent}.

%Závěr
\section*{Závěr}
Změřili jsme závislost doby kmitu proudu v sériovém RLC obvodu na kapacitě (viz tabulka \ref{t:per} a graf \ref{g:per}).
Z této závislosti jsme určili indukčnost cívky $L_1=\SI{0.52(5)}{\henry}$.

Změřili jsme závislost aperiodizačního odporu na kapacitě (viz tabulka \ref{t:aper} a graf \ref{g:aper}).
Z této závislosti jsme opět určili indukčnost též cívky $L_2=\SI{0.34(2)}{\henry}$.

Hodnoty $L_1$ a $L_2$, které vyjadřují touž veličinu měřenou různými způsoby, se od sebe liší. Jako přesnější považujeme $L_1$ (viz diskuze).

Změřili jsme závislost relaxační doby sériového RC obvodu na velikosti kapacity a odporu (viz tabulka \ref{t:RC} a grafy \ref{g:RCr} a \ref{g:RCc}). Tato závislost je podle očekávání lineární v $R$ i v $C$.


\printbibliography[title={Seznam použité literatury}]

\end{document}