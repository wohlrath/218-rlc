\section*{Teoretická část}
Pokud se v sériovém RLC obvodu změní v čase $t=0$ přivedené napětí z nulové hodnoty na hodnotu $\varepsilon$, pak má časový průběh proudu v sériovém RLC obvodu různý charakter v závislosti na hodnotách $R$, $L$ a $C$.
Označíme
\begin{equation*}
A=\frac{R}{2L} \,, \qquad  B^2=\frac{1}{LC} - \frac{R^2}{4L^2} \,.
\end{equation*}

Potom je průběh proudu podle \cite{skripta}
\begin{enumerate}
\item pro $A^2<1/LC$: periodické --- $I(t) = (\varepsilon/BL) \exp(-At) \sin(Bt)$
\item pro $A^2=1/LC$: mezní aperiodické --- $I(t) = (\varepsilon/L)t \exp(-At)$
\item pro $A^2>1/LC$: aperiodické --- $I(t) = (\varepsilon/BL) \exp(-At) \sinh(Bt)$
\end{enumerate}

Při periodickém pohybu odpovídá konstanta $B$ kruhové frekvenci kmitů.
Pokud platí $1/LC \gg (R/L)^2$, což je v našem případě dobře splněno, můžeme indukčnost $L$ vyjádřit z periody kmitů
\begin{equation} \label{e:Lper}
L=\frac{T^2}{4\pi^2 C} \,.
\end{equation}

Při mezním aperiodickém pohybu můžeme podmínku $A^2=1/LC$ upravit na $R_{ap}=2 \sqrt{L/C}$.
Pokud tedy správně určíme aperiodizační odpor $R_{ap}$, můžeme vypočítat indukčnost $L$ podle
\begin{equation} \label{e:Laper}
L=\frac{R_{ap}^2C}{4} 	,.
\end{equation}

V sériovém RC obvodu po vypnutí zdroje klesá proud s časem úměrně funkci $\exp(-t/\tau)$, kde
\begin{equation} \label{e:tau}
\tau=RC \,,
\end{equation}
je \emph{relaxační doba}.